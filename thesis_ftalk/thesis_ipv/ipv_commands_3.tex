\newcommand{\latex}{\LaTeX}
\newcommand{\biblatex}{Bib\LaTeX}
\newcommand{\tab}[1]{Tabelle \ref{tab:#1}}
\newcommand{\Egquer}{{\bar{E}_{ \mathrm{g}}}}
\newcommand{\sigmaEg}{{\sigma_{ \mathrm{Eg}}}}
\newcommand{\fig}[1]{Bild \ref{fig:#1}}
% Einheiten
\newcommand{\mJ}{\unit{mJ}}
\newcommand{\kV}{\unit{kV}}
\newcommand{\mAh}{\unit{mAh}}
\newcommand{\g}{\unit{g}}
\newcommand{\As}{\unit{As}}
\newcommand{\mol}{\unit{mol}}
\newcommand{\mA}{\unit{mA}}
\newcommand{\kB}{{k_{ \mathrm{ B}}}}
\newcommand{\m}{\unit{m}}
\newcommand{\J}{\unit{J}}
\newcommand{\mbar}{\unit{mbar}}
\newcommand{\unit}[1]{{\,\rm #1}} %Einheiten (Aufrecht)
\newcommand{\A}{\unit{A}}
\newcommand{\Celsius}{\unit{^\circ C}}
\newcommand{\cm}{\unit{cm}}
\newcommand{\dB}{\unit{dB}} %deziBel
\newcommand{\Dekade}{\unit{Dec}} %Dekade
\newcommand{\eV}{\unit{eV}}
\newcommand{\eVolt}{\unit{eV}}
\newcommand{\Farad}{\unit{F}}
\newcommand{\GHz}{\unit{GHz}}
\newcommand{\Hz}{\unit{Hz}}
\newcommand{\inch}{\unit{inch}} %Ein Zoll
\newcommand{\K}{\unit{K}}
\newcommand{\kHz}{\unit{kHz}}
\newcommand{\meV}{\unit{meV}}
\newcommand{\MHz}{\unit{MHz}}
\newcommand{\mm}{\unit{mm}}
\newcommand{\mSek}{\unit{ms}} %Millisekunde
\newcommand{\mum}{\unit{\mu m}}
\newcommand{\muSek}{\unit{\mu s}} %Mikrosekunde
\newcommand{\mTorr}{\unit{mTorr}}
\newcommand{\mV}{\unit{mV}}
\newcommand{\mVolt}{\unit{mV}}
\newcommand{\mW}{\unit{mW}}
\newcommand{\nF}{\unit{nF}} %Nanno Farrad
\newcommand{\nH}{\unit{nH}} %Nanno Henry
\newcommand{\nm}{\unit{nm}}
\newcommand{\nSek}{\unit{ns}} %Nanosekunde
\newcommand{\Ohm}{\Omega}
\newcommand{\percent}{\unit{\%}}
\newcommand{\pF}{\unit{pF}} %Pico Farrad
\newcommand{\pH}{\unit{pH}} %Pico Henry
\newcommand{\pSek}{\unit{ps}} %Picosekunde
\newcommand{\s}{\unit{s}} %Sekunde
\newcommand{\V}{\unit{V}}
\newcommand{\W}{\unit{W}}
\renewcommand{\min}{\unit{min}} %Sekunde
\newcommand{\keV}{\unit{keV}}
\newcommand{\h}{\unit{h}}
\newcommand{\sq}{\unit{sq}}
\newcommand{\pA}{\unit{pA}}
\newcommand{\nA}{\unit{nA}}
\newcommand{\Ah}{\unit{Ah}}
\newcommand{\kOhm}{\unit{k\Omega}}

% Schriftgroessen im Math-Modus
%
\newcommand{\TS}{\textstyle}
\newcommand{\DS}{\displaystyle}
\newcommand{\SCS}{\scriptstyle}
\newcommand{\SCSCS}{\scriptscriptstyle}
%
%
% Schriftarten im Math-Modus
%
\newcommand{\kuno}[1]{$#1$}                  %kursiv, normal
\newcommand{\upno}[1]{\mbox{\rm{#1}}}        %upright(senkrecht), bold
\newcommand{\kubo}[1]{\mbox{\boldmath $#1$}} %kursiv, bold
\newcommand{\upbo}[1]{\mbox{\rm\bf{#1}}}     %upright(senkrecht), bold
%
%%%%%%%%%%%%%%%%%%%%%%%%%%%%%%%%%%%%%%%%%%%%%%%%%%%%%%%%%%%%%%%%%%%%%%%
%Gleichungen und Formeln
%%%%%%%%%%%%%%%%%%%%%%%%%%%%%%%%%%%%%%%%%%%%%%%%%%%%%%%%%%%%%%%%%%%%%%%
\newcommand{\eqn}[2]{\begin{equation}
#1
\label{eqn:#2}
\end{equation}}
%%%%%%%%%%%%%%%%%%%%%%%%%%%%%%%%%%%%%%%%%%%%%%%%%%%%%%%%%%%%%%%%%%%%%%%


%***********************************************************************%
%
% Schriftgroesse fuer xfig-bilder
%
%***********************************************************************%
%
\newcommand{\tsl}[1]{\large #1}
\newcommand{\tsn}[1]{\normalsize #1}
\newcommand{\tss}[1]{\small #1}
%\newcommand{\tsl}[1]{\Large #1}
%\newcommand{\tsn}[1]{\large #1}
%\newcommand{\tss}[1]{\normalsize #1}
\newcommand{\fsl}[1]{\tsl{$#1$}}
\newcommand{\fsn}[1]{\tsn{$#1$}}
\newcommand{\fss}[1]{\tss{$#1$}}
%


%%%%%%%%%%%%%%%%%%%%%%%%%%%%%%%%%%%%%%%%%%%%%%%%%%%%%%%%%%%%%%%%%%%%%%%%%%%
%
%Tabellen nicht als Abkürzung sondern zum Rauskopieren
%
%%%%%%%%%%%%%%%%%%%%%%%%%%%%%%%%%%%%%%%%%%%%%%%%%%%%%%%%%%%%%%%%%%%%%%%%%%%

%\begin{table}[h!tbp]
%\mycaption{Verdammt wichtige physikalische Konstanten}{}
%\vspace{5mm}
%  \centering
%  \begin{tabular}{ l l }
%    \hline
%    \hline
%    % after \\: \hline or \cline{col1-col2} \cline{col3-col4} ...
%    Größe & Wert \\
%    \hline
%    $q$ &      $1.6 \times 10^{-19} \A\s $ \\
%   $\varepsilon_0 $ &  $ 8.85 \times 10^{-14} \A\s/\V\cm$ \\
%    \hline
%    \hline
%  \end{tabular}
%\vspace{5mm}
%\label{tab:physkonst}
%\end{table}


%***********************************************************************%
%
% Bilder, Rahmen
%
%***********************************************************************%
%\includeps{Breite in Zentimetern 'cm'}{bild_datei}{Unterschrift}{Eintrag ins Abbildungsverzeichnis}
\newcommand{\includeps}[4]{
\begin{figure}[h!tbp]
\vspace{3mm}
  \begin{center}
       \includegraphics[width=#1]{#2.eps}
 \mycaption{#3}{#4}
    \label{fig:#2}
  \end{center}
\vspace{5mm}
\end{figure}
}%
%% Einbinden von PDFs als Bilder
\newcommand{\includepdf}[4]{
\begin{figure}[h!tbp]
\vspace{3mm}
  \begin{center}
       \includegraphics[width=#1]{#2.pdf}
 \mycaption{#3}{#4}
    \label{fig:#2}
  \end{center}
\vspace{5mm}
\end{figure}
}%

%%%  Einbinden von svg's aus inkscape (mit Latexschrift in Bild)
\newcommand{\executeiffilenewer}[3]{%
\ifnum\pdfstrcmp{\pdffilemoddate{#1}}%
{\pdffilemoddate{#2}}>0%
{\immediate\write18{#3}}\fi%
}

\newcommand{\includesvg}[4]{
\begin{figure}[h!tbp]
\vspace{3mm}
  \begin{center}
   \def\svgwidth{#1}
   \executeiffilenewer{#2.svg}{#2.pdf}%
	{inkscape -z -D --file=#2.svg %
	--export-pdf=#2.pdf --export-latex}%
	\input{#2.pdf_tex}
	 \mycaption{#3}{#4}
	    \label{fig:#2}
	  \end{center}
	\vspace{5mm}
	\end{figure}
}
%%%Die folgenden beiden Befehle sind nur für xfig Benutzer relevant
%%Optionen zum Aufruf von xfig, damit der Export von pstex-Grafiken gut funktioniert:
%%xfig -metric -right -but_per_row 3 -portrait -zoom 2 -pheight 18 -pwidth 24 -latexfonts -specialtext -startgridmode 1 -exportL pstex
%%Oder als alias einrichten: z. B.
%%alias lafig 'xfig -metric -right -but_per_row 3 -portrait -zoom 2 -pheight 18 -pwidth 24 -latexfonts -specialtext -startgridmode 1 -exportL pstex'  # xfig fuer latex
%
%%\includepstex{bild_datei}{Unterschrift}{Eintrag ins Abbildungsverzeichnis}
%\newcommand{\includepstex}[3]{
%\begin{figure}[h!tbp]
%\vspace{5mm}
%  \begin{center}
%    %\figbox{ \input{#1.pstex_t} }
%    \input{#1.pstex_t}
%   \mycaption{#2}{#3}
%    \label{fig:#1}
%    \vspace{3mm}
%  \end{center}
%\end{figure}
%}%

%
%%\includepstexshift{bild_datei}{Unterschrift}{H-Shift}{eintrag ins Abbildungsverzeichnis}
%\newcommand{\includepstexshift}[4]{
%\begin{figure}[h!tbp]
%  \begin{center}
%    \figbox{ \hspace*{#3}\input{#1.pstex_t} }
%    \mycaption{#2}{#4}
%    \label{fig:#1}
%  \end{center}
%\end{figure}}%
