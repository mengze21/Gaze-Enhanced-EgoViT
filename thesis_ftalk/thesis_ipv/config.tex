%\documentclass[12pt,twoside,a4paper]{report}
\usepackage{acronym}
\usepackage{pdflscape} % Add this line to import the necessary package
\usepackage{a4}
%\usepackage[ngerman]{babel}    % sorgt für deutsche Überschriften, neue deutsche Rechtschreibung usw.
\usepackage[normal,bf]{caption}    % Formats table and figure captions (see Pdf-file eps-latex p. 51)
\usepackage{graphicx} % to include graphics
%\usepackage{cite}   % to order references, e.g. [12-17] (or use package citesort)
\usepackage{amsmath}    % for ams-tex symbols
\usepackage{amsfonts}
\usepackage{amssymb}
\usepackage{fancyhdr}   % header layout (see Pdf-file fancyhdr)
\usepackage{xspace}     % for empty lines in Mathmode
% \usepackage[ngerman]{babel}
\usepackage[american]{babel}
\usepackage[utf8]{inputenc}

\usepackage{floatflt}
\usepackage{xcolor} % notwendig zur Farbumwandlung für hyperref
\usepackage[pdfborder={0 0 0},citecolor={red}]{hyperref}
\usepackage{listings}
\usepackage{subfigure}
% \usepackage[caption=false, lofdepth=1, lotdepth, margin=5pt]{subfig}
% \usepackage{subcaption}
%\usepackage{subfig}
\usepackage{csquotes}
\usepackage[style=phys,backend=biber,biblabel=brackets]{biblatex}
\addbibresource{bibliography.bib}   % Pfad zu *.bib Dateien für Literaturverzeichnis
% \addbibresource{bibliography.bib}
\usepackage{url}
\urlstyle{same}

\usepackage{makeidx}    % Stichwortverzeichnis
\makeindex

\usepackage{array}

% Glossaries
\usepackage[acronym,indexonlyfirst,nomain]{glossaries}
\renewcommand*{\acronymname}{List of Abbreviations}
\renewcommand*{\glsgroupskip}{}
\makenoidxglossaries

\usepackage{csquotes}
%%%%%%%%%%%%%%%%%%%%%%%%%automatic EPS to PDF conversion(only if eps is newer than pdf%%%%%%%%%%%%%%%%%
\newif\ifpdf
\ifx\pdfoutput\undefined
   \pdffalse
\else
   \pdfoutput=1
   \pdftrue
\fi
\ifpdf
   \usepackage{epstopdf}
   \DeclareGraphicsRule{.eps}{pdf}{.pdf}{`epstopdf #1}
   \pdfcompresslevel=9
\fi
\epstopdfsetup{suffix=}
%%%%%%%%%%%%%%%%%%%%%%%%%%%%%%%%%%%%%%%%%%%%%%%%%%%%%%%%%%%%%%%%%%%%%%%%%%%%%%%%%%%%%%%%%%%%%%%%%%%%%%%%%

%%%%%%%%%%%%%%%%%%%%%%%%%%%%%%Abkürzungsverzeichnis%%%%%%%%%%%%%%%%%%%%%%%%%%%%%%%%%%%%%%%%%%%%%%%%%%%%%%%%%%%%%%%%%%%%%%%
%\usepackage[german]{nomencl} % german macht aus der Überschrift Nomenclature Symbolverzeichnis
%\usepackage{expdlist}
%\renewcommand{\nomname}{Formelzeichen und Abkürzungen}
%\newcommand{\nomunit}[1]{\renewcommand{\nomentryend}{\hspace*{\fill}#1}}
%\makenomenclature % zu Anfang auskommentieren
%Befehl ausführen mit
%(Kopieren und mit rechter Maustaste in Konsole einfügen)
%makeindex  Masterarbeit.nlo -s  nomencl.ist -o Masterarbeit.nls
%Abkürzung für \nomenclature (fz für Formelzeichen)
% \fz{Symbol}{Beschreibung}{Einheit}
%\newcommand{\fz}[3]{\nomenclature{#1}{#2 \nomunit{#3}}}
%
\newacronym{abc}{ABC}{A Basic Concept}
\newacronym{er}{ER}{error rate}
\newacronym{fr}{FR}{Fehlerrate}
\newacronym[plural={RDBMS},shortplural={RDBMS}]{rdbms}{RDBMS}{Relational Database Management System}
\newacronym{vr}{VR}{Virtual Reality}
\newacronym{hri}{HRI}{Human-Robot Interaction}
\newacronym{ear}{EAR}{egocentric action recognition}
\newacronym{vit}{ViT}{Vision Transformer}
\newacronym{dctg}{DCTG}{dynamic class token generator}
\newacronym{padm}{PADM}{Pyramid Architecture with a Dynamic Merging}
\newacronym{fpv}{FPV}{first person view}
\newacronym{hod}{HOD}{Hand and Object Detector}
\newacronym{cnn}{CNNs}{convolutional neural networks}
\newacronym{tsn}{TSN}{Temporal Segment Networks}
\newacronym{tsm}{TSM}{termed temporal shift module}
\newacronym{nlp}{NLP}{natural language processing}
\newacronym{vivit}{ViViT}{Video Vision Transformer}
\newacronym{vst}{}{Video Swin Transformer}
\newacronym{msa}{MSA}{multi-Head Self-Attention}
\newacronym{ffn}{FFN}{feed-forward network}
\newacronym{ln}{LN}{Layer Normalization}
\newacronym{dct}{DCT}{Dynamic Class Token}
\newacronym{gedctg}{GEDCTG}{Gaze-Enhanced Dynamic Class Token Generator}
\newacronym{lstm}{LSTM}{long short-term memory}
%%%%%%%%%%%%%%%%%%%%%%%%%%%%%%%%%%%%%%%%%%%%%%%%%%%%%%%%%%%%%%%%%%%%%%%%%%%%%%%%%%%%%%%%%%%%%%%%%%%%%%%%%%%%%%%%%%%%%%%

%%%%%%%%%%%%%%%%%%%%%%%%%%%%%%%%%%%%%%%%%%%
%  paper format
%%%%%%%%%%%%%%%%%%%%%%%%%%%%%%%%%%%%%%%%%%%
%-----------------------------------------
\setlength{\paperheight}{297mm}
\setlength{\paperwidth}{210mm}
\setlength{\evensidemargin}{5mm}
\setlength{\oddsidemargin}{5mm}
\setlength{\topmargin}{0mm}
\setlength{\footskip}{12mm}
%-----------------------------------------

%%%%%%%%%%%%%%%%%%%%%%%%%%%%%%%%%%%%%%%%%%
% text paragraphs
%%%%%%%%%%%%%%%%%%%%%%%%%%%%%%%%%%%%%%%%%%%
\setlength{\textwidth}{15cm}

%-----------------------------------------
% line spacing
\renewcommand{\baselinestretch}{1.2}
% paragraph indent
\setlength{\parindent}{5mm}
%-----------------------------------------
%
%%%%%%%%%%%%%%%%%%%%%%%%%%%%%%%%%%%%%%%%%%%
%  header and footer
%%%%%%%%%%%%%%%%%%%%%%%%%%%%%%%%%%%%%%%%%%%
%-----------------------------------------

%\clearpage{\pagestyle{empty}\cleardoublepage}   % empty page without header
%-----------------------------------------
%

%%%%%%%%%%%%%%%%%%%%%%%%%%%%%%%%%%%%%%%%%%
%  footnote control and header
%%%%%%%%%%%%%%%%%%%%%%%%%%%%%%%%%%%%%%%%%%%
%-----------------------------------------
\pagestyle{fancy}
\fancyhf{} % clear all fields
\fancyhead[LE,RO]{\thepage}        % E and O are Even and Odd pages, L, R and C stand for
                                    % Left, Right und Center
\fancyhead[RE]{\slshape \leftmark} % \leftmark (higher level, e.g. chapter) is first
                                    % argument of \markboth{}{},
                                    % \rightmark (lower level, e.g. section) second %argument (or  argument of \markright{})
\fancyhead[LO]{\slshape \rightmark} % \fancyfoot[LE,RO]{\thepage}
\renewcommand{\chaptermark}[1]{\markboth{\MakeUppercase{\chaptername\ \thechapter.\ #1}}{}} % \chaptername=name of chapter \thechapter=Number of chapter Argument #1 is the
        % name of the chapter
\renewcommand{\sectionmark}[1]{\markright{\MakeUppercase{\thesection\ #1}}}
\renewcommand{\headheight}{1.5cm}
\renewcommand{\headrulewidth}{0.4pt}
\renewcommand{\footrulewidth}{0.4pt}

%%%%%%%%%%%%%%%%%%%%%%%%%%%%%%%%%%%%%%%%%%%
%  settings for float environments such as figures
%%%%%%%%%%%%%%%%%%%%%%%%%%%%%%%%%%%%%%%%%%%
%-----------------------------------------
% page layout
\setcounter{topnumber}{2}     % Maximum number of floats at the top of a page
\setcounter{bottomnumber}{2}  % Maximum number of floats at the bottom of a page
\setcounter{totalnumber}{4}   % Maximum number of floats on a page
\renewcommand{\topfraction}{1}     %
\renewcommand{\bottomfraction}{1}  %
\renewcommand{\textfraction}{0.05} % minimal fraction of text on a page
\renewcommand{\floatpagefraction}{0.7}  %minimal fraction of floats on a pure float page
%-----------------------------------------
% distances to float objects
\setlength{\floatsep}{0pt}      % spaces at top and bottom of a page
\setlength{\textfloatsep}{0ex}  % spaces to text
\setlength{\intextsep}{0ex}     % spaces in text
%-----------------------------------------
%

%%%%%%%%%%%%%%%%%%%%%%%
% Table and figure captions
% see usepackage caption2
%%%%%%%%%%%%%%%%%%%%%%%
%\addto\captionsenglish{%
%\renewcommand\chaptername{Kapitel}}
%\renewcommand{\chaptername}{Kapitel}
\renewcommand{\figurename}{Bild}
\addto\captionsngerman{\renewcommand{\figurename}{Bild}}
\renewcommand{\tablename}{Tab.}




%\addto\captionsngerman{\renewcommand{\listfigurename}{Bildverzeichnis}}


%\renewcaptionname{ngerman}{\listfigurename}{Bildverzeichnis}

% \mycaption{Table or Figure caption}{entry for the list of tables / figures}
\newcommand{\mycaption}[2]{\caption[#2]{\tss{#1}}}

% Trennungshilfe
\hyphenation{Se-lek-tion Mak-ro-zel-le}


%%%%%%%%%%%%%%%%%%
% directory for figures
%%%%%%%%%%%%%%%%%%%%
\makeatletter
\def\input@path{{figures/}{figures/figures2/}{figures/grafiken/}}
\makeatother %  definiert in welchen Verzeichnissen bei Aufruf von /input gesucht werden soll, notwenig für \includesvg

\graphicspath{{figures/}{figures/figures2/}{figures/grafiken}}    % defines the path where latex looks for figures